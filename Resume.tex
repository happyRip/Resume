\PassOptionsToPackage{unicode=true}{hyperref} % options for packages loaded elsewhere
\PassOptionsToPackage{hyphens}{url}
%
\documentclass[]{article}
\usepackage{lmodern}
\usepackage{amssymb,amsmath}
\usepackage{ifxetex,ifluatex}
\usepackage{fixltx2e} % provides \textsubscript
\ifnum 0\ifxetex 1\fi\ifluatex 1\fi=0 % if pdftex
  \usepackage[T1]{fontenc}
  \usepackage[utf8]{inputenc}
  \usepackage{textcomp} % provides euro and other symbols
\else % if luatex or xelatex
  \usepackage{unicode-math}
  \defaultfontfeatures{Ligatures=TeX,Scale=MatchLowercase}
\fi
% use upquote if available, for straight quotes in verbatim environments
\IfFileExists{upquote.sty}{\usepackage{upquote}}{}
% use microtype if available
\IfFileExists{microtype.sty}{%
\usepackage[]{microtype}
\UseMicrotypeSet[protrusion]{basicmath} % disable protrusion for tt fonts
}{}
\IfFileExists{parskip.sty}{%
\usepackage{parskip}
}{% else
\setlength{\parindent}{0pt}
\setlength{\parskip}{6pt plus 2pt minus 1pt}
}
\usepackage{hyperref}
\hypersetup{
            pdfborder={0 0 0},
            breaklinks=true}
\urlstyle{same}  % don't use monospace font for urls
\usepackage[margin=1in]{geometry}
\usepackage{graphicx,grffile}
\makeatletter
\def\maxwidth{\ifdim\Gin@nat@width>\linewidth\linewidth\else\Gin@nat@width\fi}
\def\maxheight{\ifdim\Gin@nat@height>\textheight\textheight\else\Gin@nat@height\fi}
\makeatother
% Scale images if necessary, so that they will not overflow the page
% margins by default, and it is still possible to overwrite the defaults
% using explicit options in \includegraphics[width, height, ...]{}
\setkeys{Gin}{width=\maxwidth,height=\maxheight,keepaspectratio}
\setlength{\emergencystretch}{3em}  % prevent overfull lines
\providecommand{\tightlist}{%
  \setlength{\itemsep}{0pt}\setlength{\parskip}{0pt}}
\setcounter{secnumdepth}{0}
% Redefines (sub)paragraphs to behave more like sections
\ifx\paragraph\undefined\else
\let\oldparagraph\paragraph
\renewcommand{\paragraph}[1]{\oldparagraph{#1}\mbox{}}
\fi
\ifx\subparagraph\undefined\else
\let\oldsubparagraph\subparagraph
\renewcommand{\subparagraph}[1]{\oldsubparagraph{#1}\mbox{}}
\fi

% set default figure placement to htbp
\makeatletter
\def\fps@figure{htbp}
\makeatother

\newenvironment{cols}[1][]{}{}

\newenvironment{col}[1]{\begin{minipage}{#1}\ignorespaces}{%
\end{minipage}
\ifhmode\unskip\fi
\aftergroup\useignorespacesandallpars}

\def\useignorespacesandallpars#1\ignorespaces\fi{%
#1\fi\ignorespacesandallpars}

\makeatletter
\def\ignorespacesandallpars{%
  \@ifnextchar\par
    {\expandafter\ignorespacesandallpars\@gobble}%
    {}%
}
\makeatother
\defaultfontfeatures{Extension = .otf}
\usepackage{fontawesome}
\pagenumbering{gobble}
\usepackage{marvosym}

\author{}
\date{\vspace{-2.5em}}

\begin{document}

\begin{cols}

\begin{col}{0.4\textwidth}

\hypertarget{aleksander-borowski}{%
\section{Aleksander Borowski}\label{aleksander-borowski}}

\end{col}

\begin{col}{0.2\textwidth}

~

\end{col}

\begin{col}{0.4\textwidth}

\flushright{\textbf{Devops developer}}

\end{col}

\end{cols}

\begin{cols}

\begin{col}{0.4\textwidth}

\Telefon ~\emph{Given upon request}

\faHome ~Wrocław, Dolnośląskie - Poland ~\\

\end{col}

\begin{col}{0.2\textwidth}

~

\end{col}

\begin{col}{0.4\textwidth}

\Letter ~\href{mailto:aleksander.borowski.01@gmail.com}{\nolinkurl{aleksander.borowski.01@gmail.com}}

\href{https://github.com/happyRip}{\faGithub ~github.com/happyRip} ~\\

\end{col}

\end{cols}

\begin{center}\rule{0.5\linewidth}{0.5pt}\end{center}

\hypertarget{experience}{%
\subsection{Experience}\label{experience}}

\begin{cols}

\begin{col}{0.2\textwidth}

\emph{Sep 2021 - present}

\end{col}

\begin{col}{0.05\textwidth}

~

\end{col}

\begin{col}{0.75\textwidth}

\hypertarget{devops-working-student}{%
\subsubsection{DevOps Working Student}\label{devops-working-student}}

\begin{quote}
\emph{Nokia Wrocław}
\end{quote}

I'm responsble for managing and hardening embedded linux systems using
\href{https://www.yoctoproject.org/}{the~Yocto~Project} stack. Here I've
learned core Agile principles and common Jira workflow in practice. ~\\

\end{col}

\end{cols}

\begin{center}\rule{0.5\linewidth}{0.5pt}\end{center}

\hypertarget{education}{%
\subsection{Education}\label{education}}

\begin{cols}

\begin{col}{0.2\textwidth}

2019 - present

\end{col}

\begin{col}{0.05\textwidth}

~

\end{col}

\begin{col}{0.75\textwidth}

\hypertarget{electronics-and-telecommunication}{%
\subsubsection{Electronics and
Telecommunication}\label{electronics-and-telecommunication}}

\begin{quote}
\emph{Wrocław University of Technology,}

\emph{Faculty of Microsystem Electronics and Photonics} ~\\
\end{quote}

\end{col}

\end{cols}

\begin{center}\rule{0.5\linewidth}{0.5pt}\end{center}

\hypertarget{projects}{%
\subsection{Projects}\label{projects}}

\begin{cols}

\begin{col}{0.2\textwidth}

2020 - present

\hypertarget{technologies}{%
\paragraph{Technologies:}\label{technologies}}

Go \emph{(\href{https://sciter.com/}{Sciter})}

\end{col}

\begin{col}{0.05\textwidth}

~

\end{col}

\begin{col}{0.75\textwidth}

\begin{cols}

\begin{col}{0.4\textwidth}

\hypertarget{box-tailor}{%
\subsubsection{\texorpdfstring{\href{https://github.com/happyRip/Box-Tailor-Go}{\faGithub ~Box
Tailor}}{~Box Tailor}}\label{box-tailor}}

\end{col}

\begin{col}{0.05\textwidth}

~

\end{col}

\begin{col}{0.55\textwidth}

\begin{quote}
\emph{Parametric box design}
\end{quote}

\end{col}

\end{cols}

\qquad An app than can be used to create and export boxes based on
product measurements given. The templates arose are then optimally
fitted onto boards of given dimensions. This way the whole template can
then be cut on a laser cutter or cnc plotter, allowing small businesses
to have their own custom box solution. ~\\

\end{col}

\end{cols}

\begin{cols}

\begin{col}{0.2\textwidth}

2020, 2021

\hypertarget{technologies-1}{%
\paragraph{Technologies:}\label{technologies-1}}

Go \emph{(\href{https://github.com/spf13/cobra}{cobra},
\href{https://github.com/charmbracelet/bubbletea}{bubbletea})}, make

\end{col}

\begin{col}{0.05\textwidth}

~

\end{col}

\begin{col}{0.75\textwidth}

\begin{cols}

\begin{col}{0.4\textwidth}

\hypertarget{typer}{%
\subsubsection{\texorpdfstring{\href{https://github.com/maaslalani/typer}{\faGithub ~Typer}}{~Typer}}\label{typer}}

\end{col}

\begin{col}{0.05\textwidth}

~

\end{col}

\begin{col}{0.55\textwidth}

\begin{quote}
\emph{Typing test in your terminal}
\end{quote}

\end{col}

\end{cols}

\qquad Typer is a terminal application that measures your typing speed.

I've contributed to this open-source typing test implementing
functionality such as command line interface, flags, piping, as well as
bugfixes and more. ~\\

\end{col}

\end{cols}

\begin{cols}

\begin{col}{0.2\textwidth}

2021 - present

\hypertarget{technologies-2}{%
\paragraph{Technologies:}\label{technologies-2}}

rMarkdown, Actions

\end{col}

\begin{col}{0.05\textwidth}

~\\

\end{col}

\begin{col}{0.75\textwidth}

\begin{cols}

\begin{col}{0.4\textwidth}

\hypertarget{resume}{%
\subsubsection{\texorpdfstring{\href{https://github.com/happyRip/Resume}{\faGithub ~Resume}}{~Resume}}\label{resume}}

\end{col}

\begin{col}{0.05\textwidth}

~

\end{col}

\begin{col}{0.55\textwidth}

\begin{quote}
\emph{My personal resume}
\end{quote}

\end{col}

\end{cols}

\qquad The document You are seeing now was created in \emph{rMarkdown}
and is converted to a pdf format automatically using \emph{pandoc} and
\emph{GitHub Actions}. ~\\

\end{col}

\end{cols}

\begin{center}\rule{0.5\linewidth}{0.5pt}\end{center}

\begin{cols}

\begin{col}{0.47\textwidth}

\hypertarget{skills}{%
\subsection{Skills}\label{skills}}

\begin{itemize}
\tightlist
\item
  Go \emph{(test, \href{https://sciter.com/}{Sciter,}
  \href{https://github.com/spf13/cobra}{Cobra},
  \href{https://github.com/charmbracelet/bubbletea}{Bubbletea,}
  \href{https://github.com/go-chi/chi}{go-chi},
  \href{https://github.com/satori/go.uuid}{go.uuid})}
\item
  Python \emph{(numpy, pandas, matplotlib)}
\item
  git, GitHub \emph{(Actions)}, gerrit
\item
  UNIX, bash, vim, ssh, tmux, make
\item
  Docker, Kubernetes
\item
  (r)Markdown, XML, Json, Yaml
\item
  JavaScript
  \emph{(\href{https://www.freecodecamp.org/certification/happyrip/javascript-algorithms-and-data-structures}{certificate})},
  C++ \emph{(std)} ~\\
\end{itemize}

\end{col}

\begin{col}{0.13\textwidth}

~

\end{col}

\begin{col}{0.4\textwidth}

\hypertarget{languages}{%
\subsection{Languages}\label{languages}}

\begin{itemize}
\tightlist
\item
  Polish - \emph{native}
\item
  English - \emph{professional}
\item
  German - \emph{elementary}
\end{itemize}

\hypertarget{interests}{%
\subsection{Interests}\label{interests}}

\begin{itemize}
\tightlist
\item
  \href{https://en.wikipedia.org/wiki/Computer-aided_design}{Computer
  Aided Design}
\item
  Electronics and Tinkering
\item
  Graphic design, photography
\end{itemize}

\end{col}

\end{cols}

\end{document}
